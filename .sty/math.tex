\ifluatex
    \usepackage{lualatex-math}
\fi

% Define commands with stars
\usepackage{suffix}

% `physics` with redefined `\flatfrac`
\usepackage{physics}
\DeclareDocumentCommand\flatfrac{ m m }{#1\slash#2}

% -----------------------------------------------------------------------------
% Math macros

\newcommand{\opn}[1]{\operatorname{#1}}

\newcommand{\R}{\mathbb{R}}
\newcommand{\Z}{\mathbb{Z}}

\newcommand{\e}{\mathrm{e}}
\newcommand{\ii}{\mathrm{i}}
\newcommand{\ppi}{\mathrm{\pi}}

\newcommand{\E}{\opn{\mathbb{E}}}

\newcommand{\middlemid}{\mathrel{}\middle|\mathrel{}}
% Inspired by <https://tex.stackexchange.com/a/431820>

\DeclareMathOperator*{\argmax}{arg\,max}
\DeclareMathOperator*{\argmin}{arg\,min}

% Make `\overline` and `\underline` shorter so that lines for consecutive
% letters do not connect
\newcommand{\ol}[1]{{\mspace{1mu}\overline{\mspace{-1mu} #1 \mspace{-1mu}}\mspace{1mu}}}
\newcommand{\ul}[1]{{\mspace{1mu}\underline{\mspace{-1mu} #1 \mspace{-1mu}}\mspace{1mu}}}
% Inspired from <https://tex.stackexchange.com/questions/281391/how-to-make-underline-shorter>
% (see comments of the question).

% -----------------------------------------------------------------------------
% Bold math

% There is, to my knowledge, no straightforward way to typeset bold math. The
% main issue is that font commands in `unicode-math` don't nest. This is a
% workaround. Use `\mm` and `\vv` for bold italic letters. Use `\mm*` and
% `\vv*` for bold upright letters.

\ifpdftex
    % If `unicode-math` is not loaded, which presumably only happens when
    % compiling with PDFTeX:
    \newcommand{\mm}[1]{\vb*{#1}}
    \WithSuffix\newcommand\mm*[1]{\vb{#1}}
    \providecommand{\vv}{} % might already be defined by `newpxmath`
    \renewcommand{\vv}[1]{\vb*{#1}}
    \WithSuffix\newcommand\vv*[1]{\vb{#1}}
\else
    % If `unicode-math` is loaded:
    \newcommand{\mm}[1]{\symbfit{#1}}
    \WithSuffix\newcommand\mm*[1]{\symbfup{#1}}
    \newcommand{\vv}[1]{\symbfit{#1}}
    \WithSuffix\newcommand\vv*[1]{\symbfup{#1}}
\fi
