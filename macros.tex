% -----------------------------------------------------------------------------
% Fonts

% Palatino
\usepackage{newpxtext}
\usepackage{amsmath}
\usepackage{amsthm} % load before newpxmath
\usepackage{newpxmath}

% \usepackage{amsfonts}
% \usepackage{amssymb}

% -----------------------------------------------------------------------------
% Indices

% put before hyperref
% \usepackage{imakeidx}
% \makeindex[title=Index, intoc=true]

% use the following line to print the index
% \printindex

% -----------------------------------------------------------------------------
% Links

\usepackage[]{hyperref}
\hypersetup{
    allcolors = violet,
    breaklinks = true,
    colorlinks = true
}

% put this after hyperref and amsmath, but before theorem environments
\usepackage[capitalize,noabbrev]{cleveref}

% -----------------------------------------------------------------------------
% Theorem environments

% The default (plain) style sets theorems in italics, while the definition
% style will set them in Roman. Check the amsthm documentation for
% more detail.
\theoremstyle{definition}

\newtheorem{theorem}{Theorem}
% or use this to reset theorem counter for each section
% \newtheorem{theorem}{Theorem}[section]

% The starred versions produce unnumbered theorems, etc. Many of them
% are commented to avoid confusion, as they are not frequently used.
\newtheorem*{theorem*}{Theorem}

\newtheorem{lemma}[theorem]{Lemma}
\newtheorem*{lemma*}{Lemma}
\newtheorem{corollary}[theorem]{Corollary}
% \newtheorem*{corollary*}{Corollary}

\theoremstyle{definition}
\newtheorem*{claim}{Claim}
\newtheorem{definition}[theorem]{Definition}
% \newtheorem{definition*}{Definition}
\newtheorem{example}[theorem]{Example}
% \newtheorem*{example*}{Example}
\newtheorem{exercise}{Exercise}[section]
% \newtheorem*{exercise*}{Exercise}

\theoremstyle{remark}
\newtheorem*{note}{Note}
\newtheorem*{remark}{Remark}

\usepackage{enumerate}
\newenvironment{axioms}{\begin{enumerate}[\bf {Axiom} 1.]}{\end{enumerate}}
\newenvironment{hypotheses}{
    \begin{enumerate}[\bf {Hypothesis} 1.]}{\end{enumerate}}

% -----------------------------------------------------------------------------
% Math macros

\usepackage{mathtools}
\usepackage{physics}

% to define commands with stars
% \usepackage{suffix}

\newcommand{\opn}[1]{\operatorname{#1}}

\newcommand{\R}{\mathbb{R}}
\newcommand{\Z}{\mathbb{Z}}

\newcommand{\e}{\mathrm{e}}
\newcommand{\ii}{\mathrm{i}}

\newcommand{\mm}[1]{\vb*{#1}}
\renewcommand{\vv}[1]{\vb*{#1}}
% overwrites \vv defined in newpxmath

% intervals
% writing literal brackets might cause chktex false positives
\usepackage{interval}
\intervalconfig{soft open fences}

% non-standard integral signs
% \usepackage{esint}

% -----------------------------------------------------------------------------
% Other math related

% reset the equation counter for each section
% \numberwithin{equation}{section}

% to label systems of equations
% \usepackage{empheq}

% -----------------------------------------------------------------------------
% Figures, tables, code listings, etc.

\usepackage[dvipsnames]{xcolor}

% subfigures
\usepackage{caption}
\usepackage{subcaption}

% figures
% It is recommended to create separate files for figures.
% \usepackage{tikz}
% \usepackage{pgfplots}
% \pgfplotsset{compat=1.18}

% pseudocode
% \usepackage[noend]{algpseudocode}
% \usepackage{algorithm}

% tables
% see also https://nick-shm.gitlab.io/notes-snippets/latex/graph/tables/
\usepackage{tabularx}
\usepackage{multirow}

% -----------------------------------------------------------------------------
% Bibliography

% APA
% \usepackage[backend=biber, style=apa]{biblatex}

% Chicago (author-date)
% \usepackage[authordate, backend=biber]{biblatex-chicago}

% usage
% add to preamble:
% \addbibresource{file.bib}
% print bibliography:
% \printbibliography[heading=bibintoc]{}

% The following lines make \citeyear and \citeyearpar links. Usually
% \textcite{} is used. However, this is needed to typeset `name's (year)'.

% \DeclareCiteCommand{\citeyear}
% {}{\bibhyperref{\printfield{year}}}{\multicitedelim}{}
% \DeclareCiteCommand{\citeyearpar}
% {}{\mkbibparens{\bibhyperref{\printfield{year}}}}{\multicitedelim}{}

% -----------------------------------------------------------------------------
% Miscellaneous

% make all sections unnumbered
% \setcounter{secnumdepth}{0}

% to control spacing
\usepackage{setspace}

% smart quotation marks
\usepackage[maxlevel=4]{csquotes}
