% Additional math packages
\usepackage{lualatex-math}
\usepackage{physics}

% -----------------------------------------------------------------------------
% Math macros

% Define commands with stars
\usepackage{suffix}

\newcommand{\opn}[1]{\operatorname{#1}}

\newcommand{\R}{\mathbb{R}}
\newcommand{\Z}{\mathbb{Z}}

\newcommand{\e}{\mathrm{e}}
\newcommand{\ii}{\mathrm{i}}
\newcommand{\ppi}{\mathrm{\pi}}

\DeclareMathOperator*{\argmax}{arg\,max}
\DeclareMathOperator*{\argmin}{arg\,min}

% Makes `\overline` and `\underline` shorter so that lines for consecutive
% letters do not connect
\newcommand{\ol}[1]{{\mspace{1mu}\overline{\mspace{-1mu} #1 \mspace{-1mu}}\mspace{1mu}}}
\newcommand{\ul}[1]{{\mspace{1mu}\underline{\mspace{-1mu} #1 \mspace{-1mu}}\mspace{1mu}}}
% Inspired from <https://tex.stackexchange.com/questions/281391/how-to-make-underline-shorter>
% (see comments of the question).

% -----------------------------------------------------------------------------
% Bold math

% To my knowledge, there is no straightforward way to typeset bold math. The
% main problem is that font commands in `unicode-math` do not nest.

% The following is a workaround. Use `\mm` and `\vv` for bold italic letters. Use `\mm*`
% and `\vv*` for bold upright letters.
\newcommand{\mm}[1]{\symbfit{#1}}
\WithSuffix\newcommand\mm*[1]{\symbfup{#1}}
\newcommand{\vv}[1]{\symbfit{#1}}
\WithSuffix\newcommand\vv*[1]{\symbfup{#1}}

% Use the following if `unicode-math` is not loaded.
% \newcommand{\mm}[1]{\vb*{#1}}
% \WithSuffix\newcommand\mm*[1]{\vb{#1}}
% \newcommand{\vv}[1]{\vb*{#1}} % use `\renewcommand` if `newpxmath` is used
% \WithSuffix\newcommand\vv*[1]{\vb{#1}}
